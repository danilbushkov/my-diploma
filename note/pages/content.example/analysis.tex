
\section{Анализ предметной области}

Машина времени является гипотетическим устройством, способным перемещаться во времени, обеспечивая возможность путешествия в прошлое или будущее. В связи с этим, анализ предметной области машины времени включает следующие аспекты:


\begin{enumerate}
	\item физические принципы: необходимо изучить теоретические основы, согласно которым могла бы функционировать машина времени. Это может включать обсуждение специальной теории относительности, черных дыр, петель времени и других концепций из физики;

	\item технические особенности: рассмотрим возможные способы построения и дизайна машины времени. Какие технологии или материалы могут быть использованы для ее создания? Какие опасности или препятствия могут возникнуть при ее конструировании?

	\item парадоксы времени: проанализируем различные парадоксы, которые могут возникнуть при использовании машины времени, такие как "парадокс дедушки" или "парадокс самовыполнения". Какие последствия они могут иметь для временных путешественников?

	\item этические и социальные аспекты: обсудим влияние машины времени на общество и индивидуума. Какие этические вопросы возникают при использовании данной технологии? Какие последствия она может иметь для личной и мировой истории?

	\item возможные приложения: рассмотрим потенциальные области применения машины времени, такие как исследования истории, изменение прошлого или будущего, предсказание событий и т. д.
\end{enumerate}

В таблице \ref{t:comp-an} представлено сравнение аналагов.

\begin{table}[ht]
	\Large
	\caption{Сравнение аналогов}
	\label{t:comp-an}
	\centering
	\begin{tabularx}{\textwidth}{|X|c|c|c|}
		\hline
		Критерии \textbackslash\ Аналоги & Машина времени 1 & Машина времени 2 & Машина времени 3 \\
		\hline
		Критерий 1
		                                 & нет              & да               & да               \\
		\hline
		Критерий 2
		                                 & нет              & да               & да               \\
		\hline
		Критерий 3
		                                 & да               & да               & нет              \\
		\hline
		Критерий 4
		                                 & нет              & нет              & да               \\
		\hline
	\end{tabularx}
\end{table}

\pagebreak

\section*{Выводы}


Анализ предметной области машины времени требует не только фантазии и творческого мышления, но и глубоких знаний в области физики, философии, этики и технологии.


