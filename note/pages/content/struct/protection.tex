
\subsubsection{Обеспечение защиты информации клиентов конструктора}

Для обеспечения защищенного хранения паролей в базе данных используется
адаптивная криптографическая хеш-функция bcrypt.

Функция основана на шифре Blowfish.
Для защиты от атак с помощью радужных таблиц bcrypt использует соль (salt);
кроме того, функция является адаптивной,
время её работы легко настраивается и её можно замедлить, чтобы усложнить атаку перебором.

Функция принимает три параметра: стоимость(cost), соль и пароль.
Алгоритм хеширования состоит из следующих шагов:
\begin{enumerate}
	\item
	      инициализация состояния Blowfish с помощью дорогостоящего алгоритма настройки ключей;
	      результат состоит из массива P, состоящего из 18 подключей, и четырех S-боксов;
	\item шифрование текста «OrpheanBeholderScryDoubt» 64 раза с помощью стандартного
	      Blowfish в режиме простой замены;
	\item результат состоит из конкатенации стоимости, соли и зашифрованного текста
	      «OrpheanBeholderScryDoubt».

\end{enumerate}

Дорогостоящий алгоритм настройки ключей включает в себя следующие шаги:
\begin{enumerate}
	\item инициализация подключей P и S-боксов шестнадцатеричными цифрами числа $\pi$;
	\item перестановка P и S на основе пароля и соли: ExpandKey(P, S, password, salt);
	\item повторение $2^{cost}$ раз следующих шагов:
	      \begin{enumerate}
		      \item перестановка P и S на основе пароля: ExpandKey(P, S, password, 0);
		      \item перестановка P и S на основе соли: ExpandKey(P, S, salt, 0).
	      \end{enumerate}
\end{enumerate}

Алгоритм ExpendKey:
\begin{enumerate}
	\item смешивание пароля с массивом подключей P;
	\item
	      разбиение соли на две равные части;
	\item инициализация буфера для хранения блоков;
	\item циклическое смешивание внутреннего состояния с P, используя половины соли;
	\item циклическое смешивание зашифрованного состояния с внутренними S-блоками состояния,
	      используя половины соли.
\end{enumerate}

Аутентификация пользователя происходит по паролю и логину.
Отправленный пароль после хеширования
сравнивается с хешем из базы данных, и в случае успеха генерируется токен доступа.
Этот токен временно сохраняется в базе данных, а его копия выдается пользователю.
Используя токен, пользователь может получать доступ к функциям сервиса ботов.

