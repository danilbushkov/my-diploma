\csection{Введение}

В современном мире стали популярными такие приложения для
быстрого общения как мессенджеры. Таких приложений достаточно много, но
большинство пользователей сети интернет все чаще отдают предпочтение
мессенджеру Telegram как наиболее удобному и надежному.

У Telegram имеется удобное API для создания ботов. Бот способен
выполнять определенные команды, заданные пользователем через интерфейс
Telegram. Данный функционал вполне может удовлетворять потребности
компании в предоставлении разных услуг во многих сферах: от спортивных залов
до доставки различных товаров.

Для создания ботов требуются высококвалифицированные программисты, которые
обладают знаниями в одном или нескольких языках программирования.
Наём таких сотрудников сопровождается довольно большими тратами для бизнеса.
Также написание ботов с нуля - затратный по времени процесс.

Конструкторы или low-code решения предоставляют возможность создавать
приложения с использование визуальных блоков, что значительно упрощает процесс
разработки. Эти инструменты предлагают графические интерфейсы, шаблоны и
компоненты, которые облегчают создание приложений без необходимости писать
много кода.
Данные решения также существуют и в сфере Telegram-ботов: они
предоставляют функции создания, редактирования и управления ботами.

К сожалению, большинство таких конструкторов предоставляют ограниченный функционал
при бесплатном использовании, а также имеют закрытые
способы хранения данных клиентов. Поэтому было принято решение выполнить анализ и
разработать конструктор Telegram-ботов без данных недостатков.


\pagebreak


