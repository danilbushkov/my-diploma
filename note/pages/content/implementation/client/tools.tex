
\subsubsection{Выбор инструментов разработки}

В браузере доступны следующие инструменты для создания веб-приложения:
\begin{itemize}
	\item HTML;
	\item CSS;
	\item JavaScript.
\end{itemize}

HTML (HyperText Markup Language — «язык гипертекстовой разметки»)
— самый базовый строительный блок Веба. Он определяет содержание и
структуру веб-контента \refref{ref:html}.

Cascading Style Sheets (CSS) — это язык иерархических правил,
используемый для представления внешнего вида документа, написанного на
HTML. CSS описывает, каким образом элемент должен отображаться на
экране, на бумаге, голосом или с использованием других медиа средств \refref{ref:css}.

JavaScript — это легковесный, интерпретируемый или JIT-
компилируемый, объектно-ориентированный язык с функциями первого
класса. Наиболее широкое применение находит как язык сценариев веб-
страниц. JavaScript это прототипно-ориентированный, мультипарадигменный
язык с динамической типизацией, который поддерживает объектно-
ориентированный, императивный и декларативный (например,
функциональное программирование) стили программирования \refref{ref:js}.

Основной недостаток JavaScript – это динамическая типизация, при
которой переменная связывается с типом в момент присваивания значения, а
не в момент объявления переменной. Это особенность JS приводит к
достаточно долгой отладке кода при возникновении ошибки.

Для решения этой проблемы был выбран язык TypeScript –
компилируемый язык с возможностью явного статического назначения типов.
Статическая типизация устраняет основной недостаток JS, а компиляция
позволяет выявить некоторые ошибки до запуска приложения. Также он
совместим с JS \refref{ref:ts}.

Среди инструментов, облегчающих разработку клиентских приложений
в браузере, имеются:

\begin{itemize}
	\item Vue - JavaScript-фреймворк с открытым исходным кодом для
	      создания пользовательских интерфейсов. Легко интегрируется в проекты с
	      использованием других JavaScript-библиотек. Может функционировать как
	      веб-фреймворк для разработки одностраничных приложений в реактивном
	      стиле.
	\item Angular - открытая и свободная платформа для разработки веб-
	      приложений, написанная на языке TypeScript, разрабатываемая командой из
	      компании Google, а также сообществом разработчиков из различных
	      компаний.
	\item React - JavaScript-библиотека с открытым исходным кодом для
	      разработки пользовательских интерфейсов. React разрабатывается и
	      поддерживается Facebook, Instagram и сообществом отдельных разработчиков
	      и корпораций. React может использоваться для разработки одностраничных и
	      мобильных приложений. Его цель — предоставить высокую скорость
	      разработки, простоту и масштабируемость.
	\item SolidJS – это легковесная JavaScript библиотека для создания
	      пользовательских интерфейсов. SolidJS вдохновлен библиотекой React, но
	      нацелен на предоставление более простой и эффективной модели
	      программирования.
\end{itemize}

В качестве сравнения данных библиотек и фреймворков выделим
следующие критерии:
\begin{itemize}
	\item поддержка реактивности;
	\item сложность изучения;
	\item производительность;
	\item потребление памяти.
\end{itemize}

Под реактивностью понимается обновления отображения при
изменении привязанных данных.

Angular является фреймворком для создания крупных браузерных
решений. Он имеет довольно много возможностей и из-за этого имеет
довольно большой размер и сложен для обучения.

Vue является фреймворком для создания реактивных приложений.
Имеет много оптимизаций в своей основе: кэширование вычисляемых
свойств, умная перерисовка – перерисовываются только нужные узлы, что
позволяет работать приложению довольно быстро. Также легок в обучении и
не требует много памяти.

React является библиотекой для создания реактивных приложения.
Имеет немного низкую производительность, чем Vue, но обладает высокой
скоростью обучения и также предрасположен к низкому потреблению памяти.

SolidJS также является библиотекой для создания реактивных
приложений. Реактивность реализуется без использования виртуального
DOM, что увеличивает производительность библиотеки и уменьшает
потребление памяти.

Сравнение фреймворков и библиотек по критериям представлено в
таблице~\ref{t:comp-client-lib}.


\begin{table}[ht]
	\Large
	\begin{threeparttable}
		\caption{Сравнение фреймворков и библиотек JavaScript}
		\label{t:comp-client-lib}
		\centering
		\begin{tabularx}{\textwidth}
			{|>{}X
			|>{\centering\arraybackslash}X
			|>{\centering\arraybackslash}X
			|>{\centering\arraybackslash}X
			|>{\centering\arraybackslash}X|}
			\hline
			        &
			Vue.js  &
			React   &
			Angular &
			SolidJS                                         \\
			\hline
			Поддержка
			реактивности
			        & есть    & есть    & есть    & есть    \\
			\hline
			Сложность изучения
			        & низкая  & низкая  & высокая & низкая  \\
			\hline
			Про\-из\-во\-ди\-тель\-ность
			        & средняя & средняя & средняя & высокая \\
			\hline
			Потребление памяти
			        & низкое  & низкое  & высокое & низкое  \\
			\hline
		\end{tabularx}
	\end{threeparttable}
	\vspace{\bottompaddingoftable}
\end{table}



Исходя из таблицы можно сделать вывод, что SolidJS является самым
оптимальным выбором для разработки клиентской части для конструктора
Telegram-ботов.
