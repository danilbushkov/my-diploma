

\subsection{Структуры передаваемых данных}

Для взаимодействия серверной и клиентской частей конструктора были определены
следующие структуры:
\begin{itemize}
	\item данные пользователя;
	\item данные бота;
	\item данные компонента;
	\item данные точек соединения.
\end{itemize}

Структура данных пользователей представлена на рисунке~\ref{f:user-data-struct-code}.

\begin{figure}[ht]
	\centering
	\vspace{\toppaddingoffigure}
	\begin{lstlisting}
type UserData {
    "login": "string";
    "password": "string";
}
    \end{lstlisting}
	\caption{Структура пользователя}
	\label{f:user-data-struct-code}
\end{figure}


Структура пользователя содержит два поля: логин и пароль, которые
используются для входа в систему.

Данная структура используется при отправке данных пользователя
серверу в случае авторизации и регистрации пользователя.

Структуры данных бота представлена на рисунке~\ref{f:bot-data-struct-code}.

\begin{figure}[ht]
	\centering
	\vspace{\toppaddingoffigure}
	\begin{lstlisting}
type BotData {
    "id": "number";
    "title": "string";
    "status": "number";
}

type BotToken {
    "token": "string";
}
    \end{lstlisting}
	\caption{Структуры данных бота}
	\label{f:bot-data-struct-code}
\end{figure}

Структура данных бота содержит следующие поля:
\begin{itemize}
	\item идентификатор;
	\item название;
	\item статус.
\end{itemize}

Статус бота содержит информацию о текущем состоянии бота – бот
запущен или остановлен.

Также у бота присутствует структура его токена с одним строковым
полем.

Структура данных компонента бота представлена на
рисунке~\ref{f:component-data-struct-code}

\begin{figure}[ht]
	\centering
	\vspace{\toppaddingoffigure}
	\begin{lstlisting}
type ComponentData {
    "id": "number",
    "type": "string",
    "path": "string",
    "position": "Position";
    "connectionPoints": 
}
    \end{lstlisting}
	\caption{Структуры данных бота}
	\label{f:component-data-struct-code}
\end{figure}

Путь компонента указывает место сохранения данных после получения
результата его выполнения.

Позиция содержит два поля – поле размещения по горизонтали и
вертикали. Требуется для хранения места размещения компонента в области
редактора.

Выходы компонента содержит ассоциативный массив, где ключами
являются названия выходных точке, а значения – id следующего компонента.

Поле данных хранит данные компонента. Содержимое индивидуально
для каждого типа компонента.

Соединительная точка является входом компонента, их структура
приведена на рисунке 15.

Структура соединительных точек содержит следующие поля:

\begin{itemize}
	\item id предыдущего компонента;
	\item название выходной точки предыдущего компонента;
	\item расположение точки относительно границы компонента
\end{itemize}
