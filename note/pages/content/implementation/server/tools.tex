
\subsubsection{Выбор инструментов разработки}

В данном подразделе обосновывается выбор языка программирования
для реализации серверной части конструктора, а также системы управления
базами данных.


\paragraph{Выбор языка программирования}

Для разработки сервера в большинстве случаев выбирают следующие
языки:

\begin{itemize}
	\item C++;
	\item JavaScript;
	\item PHP;
	\item Python;
	\item Golang.
\end{itemize}

Для выбора языка, на котором будет написан сервер, выделим
следующие критерии:

\begin{itemize}
	\item удобство написания кода;
	\item быстрота исполнения кода;
	\item возможность распараллеливания программ;
	\item доступность библиотек.
\end{itemize}

C++ - компилируемый, статически типизированный язык
программирования. Имеет средства ручного управления памятью, что
позволяет достигать высокой скорости работы программы, но при этом
усложняется написание кода. Имеет средства для создания потоков, что также
может ускорить обработку запросов. Не имеет единой инфраструктуры для
управления зависимости приложения.

JavaScript – интерпретируемый, динамически типизированный язык
программирования. Благодаря платформе Node.js способен выполняться на
стороне сервера. Однопоточен, распараллеливание достигается путем запуска
нескольких экземпляров приложения. Имеет возможность асинхронного
выполнения кода. Платформа node.js предоставляет удобный пакетный
менеджер – npm. Из-за интерпретации уступает по скорости C++ и другим
компилируемым языкам.

PHP – популярный для веб-разработки язык программирования, как и
JavaScript является интерпретируемым с динамической типизацией. По
скорости также уступает компилируемым языкам, однопоточен. Имеет
популярный в кругах языка пакетный менеджер Composer.

Python – довольно популярный и простой в изучении язык
программирования, интерпретируемый с динамической типизацией. Имеет
пакетный менеджер pip, с помощью которого можно удобно управлять
зависимостями приложения. Как и другие рассмотренные интерпретируемые
языки программирования однопоточен и уступает по скорости выполнения
компилируемым языкам.

Golang – компилируемый язык программирования со статической
типизацией. Имеет единую инфраструктуру библиотек и встроенный
пакетный менеджер. Несмотря на наличие сборщика мусора, обладает
относительно высокой скоростью исполнения. Прост в изучении, имеет
простой синтаксис: не перегружен языковыми конструкциями,
минималистичен. В языке присутствуют горутины - легковесные потоки.

Сравнение языков приведено в таблице~\ref{t:comp-lang}.

\begin{table}[ht]
	\Large
	\begin{threeparttable}
		\caption{Сравнение языков программирования}
		\label{t:comp-lang}
		\centering
		\begin{tabularx}{\textwidth}
			{|>{}X
			|>{\centering\arraybackslash}X
			|>{\centering\arraybackslash}X
			|>{\centering\arraybackslash}X
			|>{\centering\arraybackslash}X|}
			\hline
			                                        &
			Удобство написания кода                 &
			Возможность распараллеливания программы &
			Быстрота исполнения кода                &
			Доступность библиотек                                                        \\
			\hline
			C++
			                                        & низкая  & есть & высокая & низкая  \\
			\hline
			JavaScript
			                                        & средняя & нет  & средняя & высокая \\
			\hline
			PHP
			                                        & средняя & нет  & средняя & высокая \\
			\hline
			Python
			                                        & высокая & нет  & средняя & высокая \\
			\hline
			Golang
			                                        & высокая & есть & высокая & высокая \\
			\hline
		\end{tabularx}
	\end{threeparttable}
	\vspace{\bottompaddingoftable}
\end{table}


По таблице можно сделать вывод, что самым оптимальным выбором для
написания веб-сервера является язык программирования Golang.

\paragraph{Выбор системы управления базами данных}

Для хранения и управления данными приложения существует множество СУБД,
но наиболее популярны такие, как PostgreSQL, MySQL и SQLServer.

SQLServer коммерческая система управления базами данных, которая предоставляет
расширенные возможности только при её покупке, что не совсем подходит под
условия создания приложения.

PostgreSQL и MySQL – это две популярные реляционные системы управления базами данных с
открытым исходным кодом. Ниже приведено сравнение этих двух баз данных
по некоторым основным критериям:
\begin{itemize}
	\item тип данных: обе СУБД поддерживают широкий спектр
	      типов данных, включая текстовые, числовые, даты и времена, бинарные и
	      другие. Однако PostgreSQL имеет более богатый набор типов данных и
	      поддерживает пользовательские типы данных;

	\item модель данных: обе СУБД используют реляционную
	      модель данных, но PostgreSQL поддерживает более сложные структуры
	      данных, включая хранимые процедуры, триггеры, пользовательские функции
	      и т.д;

	\item производительность: обе СУБД имеют хорошую
	      производительность, но в некоторых случаях PostgreSQL может быть более
	      производительным, особенно при работе с большими объемами данных;

	\item масштабируемость: обе СУБД поддерживают
	      горизонтальное и вертикальное масштабирование, но PostgreSQL обычно
	      считается более масштабируемым и подходит для крупных и сложных
	      проектов;

	\item надежность и безопасность: обе СУБД обеспечивают
	      надежность и безопасность данных, но PostgreSQL обычно обладает более
	      развитыми средствами для обеспечения защиты данных.
\end{itemize}

Таким образом выбор был сделан в пользу PostgreSQL из-за хорошей
производительности, надежности и масштабируемости.

Также для создания серверной части требуется выбрать систему
кэширования данных. Наиболее популярны такие СУБД для данных
целей, как Memcached и Redis.

Основное различие между Memcached и Redis заключается в их
функциональности. Memcached предназначен исключительно для
кэширования данных, в то время как Redis обладает более широкими
возможностями, такими как возможность работы с различными типами
данных, публикация/подписка сообщений, транзакции и другие, поэтому
выбор пал именно на него.
