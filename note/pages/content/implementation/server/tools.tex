
\subsubsection{Выбор инструментов разработки}

\paragraph{Выбор языка программирования}

Для разработки сервера в большинстве случаев выбирают следующие
языки:

\begin{itemize}
	\item C++;
	\item JavaScript;
	\item PHP;
	\item Python;
	\item Golang.
\end{itemize}

Для выбора языка, на котором будет написан сервер, выделим
следующие критерии:

\begin{itemize}
	\item удобство написания кода;
	\item быстрота исполнения кода;
	\item возможность распараллеливания программ;
	\item доступность библиотек.
\end{itemize}

C++ - компилируемый, статически типизированный язык
программирования. Имеет средства ручного управления памятью, что
позволяет достигать высокой скорости работы программы, но при этом
усложняется написание кода. Имеет средства для создания потоков, что также
может ускорить обработку запросов. Не имеет единой инфраструктуры для
управления зависимости приложения.

JavaScript – интерпретируемый, динамически типизированный язык
программирования. Благодаря платформе Node.js способен выполняться на
стороне сервера. Однопоточен, распараллеливание достигается путем запуска
нескольких экземпляров приложения. Имеет возможность асинхронного
выполнения кода. Платформа node.js предоставляет удобный пакетный
менеджер – npm. Из-за интерпретации уступает по скорости C++ и другим
компилируемым языкам.

PHP – популярный для веб-разработки язык программирования, как и
JavaScript является интерпретируемым с динамической типизацией. По
скорости также уступает компилируемым языкам, однопоточен. Имеет
популярный в кругах языка пакетный менеджер Composer.

Python – довольно популярный и простой в изучении язык
программирования, интерпретируемый с динамической типизацией. Имеет
пакетный менеджер pip, с помощью которого можно удобно управлять
зависимостями приложения. Как и другие рассмотренные интерпретируемые
языки программирования однопоточен и уступает по скорости выполнения
компилируемым языкам.

Golang – компилируемый язык программирования со статической
типизацией. Имеет единую инфраструктуру библиотек и встроенный
пакетный менеджер. Несмотря на наличие сборщика мусора, обладает
относительно высокой скоростью исполнения. Прост в изучении, имеет
простой синтаксис: не перегружен языковыми конструкциями,
минималистичен. В языке присутствуют горутины - легковесные потоки.

Сравнение языков приведено в таблице~\ref{t:comp-lang}.

\begin{table}[ht]
	\Large
	\caption{Сравнение языков программирования}
	\label{t:comp-lang}
	\centering
	\begin{tabularx}{\textwidth}
		{|>{}X
		|>{\centering\arraybackslash}X
		|>{\centering\arraybackslash}X
		|>{\centering\arraybackslash}X
		|>{\centering\arraybackslash}X|}
		\hline
		                                        &
		Удобство написания кода                 &
		Возможность распараллеливания программы &
		Быстрота исполнения кода                &
		Доступность библиотек                                                        \\
		\hline
		C++
		                                        & низкая  & есть & высокая & низкая  \\
		\hline
		JavaScript
		                                        & средняя & нет  & средняя & высокая \\
		\hline
		PHP
		                                        & средняя & нет  & средняя & высокая \\
		\hline
		Python
		                                        & высокая & нет  & средняя & высокая \\
		\hline
		Golang
		                                        & высокая & есть & высокая & высокая \\
		\hline
	\end{tabularx}
\end{table}


По таблице можно сделать вывод, что самым оптимальным выбором для
написания веб-сервера является язык программирования Golang.


\paragraph{Выбор базы данных}

Для хранения данных приложения существует множество баз данных,
но наиболее популярны такие, как PostgreSQL, MySQL и SQLServer.

SQLServer коммерческая база данных, которая предоставляет
расширенные возможности только при её покупке, что не совсем подходит под
условия создания приложения.

PostgreSQL и MySQL – это две популярные реляционные базы данных с
открытым исходным кодом. Ниже приведено сравнение этих двух баз данных
по некоторым основным критериям:





