\subsubsection{Реализация сервисов}

Сервисы предоставляют API для обеспечения выполнения основных функций конструктора.
Далее приводится описание их программных интерфейсов.

\paragraph{Коды ответов}

Сервисы работают по протоколу HTTP и имею следующие возможные
коды ответов \refref{ref:http}:
\begin{itemize}
	\item 200 (Ok) - Успешный ответ с содержимим в теле;
	\item 201 (Created) - Успешный ответ после создания ресурса. Может
	      содержать информацию о вновь созданном ресурсе;
	\item 204 (No Content) - Успешный ответ без содержимого в теле;
	\item 400 (Вad Request) - Неверно составлен запрос;
	\item 401 (Unauthorized) - Не авторизован;
	\item 404 (Not Found) - Ресурс не найден;
	\item 422 (Unprocessable Entity) - Ошибки бизнес логики. В теле ответа
	      содержится подробное описание ошибки;
	\item 500 (Internal Server Error) - Внутренняя ошибка сервера.
\end{itemize}

Формат ошибки в теле ответа, если код равен 422, представлена на
рисунке~\ref{f:error-struct}.


\begin{figure}[!ht]
	\centering
	\vspace{\toppaddingoffigure}
	\begin{lstlisting}
{
    code: integer,
    message: string
}
    \end{lstlisting}
	\caption{Формат ошибки в случае кода ответа 422}
	\label{f:error-struct}
\end{figure}

Формат ошибки содержит два значения – кода ошибки и сообщения,
которое описывает ошибку.

\paragraph{Программный интерфейс сервиса пользователей}

Сервис пользователей предоставляет следующие запросы:
\begin{itemize}
	\item регистрация:

	      Запрос: POST /api/users/signup.

	      Структура тела запроса представлена на рисунке~\ref{f:signup-request-body}.

	      \begin{figure}[ht]
		      \centering
		      \vspace{\toppaddingoffigure}
		      \begin{lstlisting}
{
    "login": "string"
    "password": "string"
}
    \end{lstlisting}
		      \caption{Тело запроса регистрации}
		      \label{f:signup-request-body}
	      \end{figure}

	      Ответ: в случае успеха - статус кода HTTP - 201. В случае некорректных
	      данных - статус кода HTTP - 422 с описанием ошибки в теле ответа;


	\item аутентификация:

	      Запрос: POST /api/users/signin.

	      Структура тела запроса представлена на рисунке~\ref{f:signup-request-body}.

	      Ответ: В случае успеха - статус кода HTTP - 201 c телом ответа, который
	      представлена на рисунке~\ref{f:signin-response-body}.

	      \begin{figure}[ht]
		      \centering
		      \vspace{\toppaddingoffigure}
		      \begin{lstlisting}
{
    "token": "string"
}
        \end{lstlisting}
		      \caption{Тело ответа при аутентификации в случае успеха}
		      \label{f:signin-response-body}
	      \end{figure}

	      В случае некорректных данных - статус кода HTTP - 422 с описание
	      ошибки в теле ответа;

	\item выход из системы:

	      Запрос: DELETE /api/users/signout.

\end{itemize}
Ответ: В случае успеха - статус кода HTTP - 204. Если пользователь не
авторизован - 401.

\paragraph{Программный интерфейс сервиса ботов}

Сервис ботов предоставляет следующий интерфейс:
\begin{itemize}
	\item создание бота:

	      Запрос: POST /api/bots.

	      Тело запроса представлено на рисунке~\ref{f:add-bot-request-body}.

	      \begin{figure}[!ht]
		      \centering
		      \vspace{\toppaddingoffigure}
		      \begin{lstlisting}
{
    "token": "string"
}
    \end{lstlisting}
		      \caption{Тело запроса при создании бота}
		      \label{f:add-bot-request-body}
	      \end{figure}

	      Код ответа в случае успеха – 201, тело ответа представлено на
	      рисунке~\ref{f:add-bot-response-body};

	      \begin{figure}[!ht]
		      \centering
		      \vspace{\toppaddingoffigure}
		      \begin{lstlisting}
{
    "botId": "integer",
    "component": "component"
}
    \end{lstlisting}
		      \caption{Тело ответа при создании бота}
		      \label{f:add-bot-response-body}
	      \end{figure}

	\item получение списка ботов:

	      Запрос: GET /api/bots.

	      В случае успеха код ответа 200, с телом ответа, содержащим список
	      ботов, структура которых представлена
	      на рисунке~\ref{f:get-bot-response-body};


	      \begin{figure}[!ht]
		      \centering
		      \vspace{\toppaddingoffigure}
		      \begin{lstlisting}
{
    "id": "integer",
    "title": "string",
    "status": "integer"
}
    \end{lstlisting}
		      \caption{Структура бота}
		      \label{f:get-bot-response-body}
	      \end{figure}

	\item удаление бота:

	      Запрос: DELETE /api/bots/{botId}.

	      В случае успеха код ответа 204;

	\item установка токена бота:

	      Запрос: POST /api/bots/{botId}/token.

	      Тело запроса представлено на рисунке~\ref{f:set-token-request-body}.

	      В случае успеха код ответа 204;


	      \begin{figure}[!ht]
		      \centering
		      \vspace{\toppaddingoffigure}
		      \begin{lstlisting}
{
    "token": "string"
}
    \end{lstlisting}
		      \caption{Тело запроса при установке токена бота}
		      \label{f:set-token-request-body}
	      \end{figure}


	\item получение токена бота:
	      Запрос: GET /api/bots/{id}/token.

	      В случае успеха код ответа 200 с телом ответа, содержащим токен бота
	      в виде строки;

	\item запуск бота:

	      Запрос: PATCH /api/bots/{id}/start.

	      В случае успеха код ответа 204;

	\item остановка бота:

	      Запрос: PATCH /api/bots/{id}/stop.

	      В случае успеха код ответа 204;

	\item получение компонентов бота:

	      Запрос: GET /api/bots/{botId}/groups/{groupId}/components

	      В случае успеха код ответа 200 с телом ответа, содержащим список
	      компонентов, структура которых представлена
	      на рисунке~\ref{f:component-struct-in-response};


	      \begin{figure}[!ht]
		      \centering
		      \vspace{\toppaddingoffigure}
		      \begin{lstlisting}[
        xleftmargin=.1\textwidth,
        xrightmargin=.1\textwidth
    ]
{
    "id": "integer",
    "type": "string",
    "data": "object",
    "path": "string",
    "outputs": {
        "nextComponentId": "integer",
        ...
    },
    "connectionPoints": {
        "<pointId: string>": {
            "sourceComponentId": "integer",
            "sourcePointName": "string",
            "relativePointPosition": {
                "x": "integer",
                "y": "integer"
            }
        }
        ...
    },
    "position": {
        "x": "integer",
        "y": "integer"
    }
}
    \end{lstlisting}
		      \caption{Структура компонента при ответе на запрос}
		      \label{f:component-struct-in-response}
	      \end{figure}


	\item создание компонента:

	      Запрос: POST /api/bots/{botId}/groups/{groupId}/components.

	      Тело запроса представлено на рисунке~\ref{f:add-component-request-body}.

	      \begin{figure}[!ht]
		      \centering
		      \vspace{\toppaddingoffigure}
		      \begin{lstlisting}
{
    "type": "string",
    "position": {
        "x": "integer",
        "y": "integer"
    }
}
    \end{lstlisting}
		      \caption{Тело запроса при создании компонента}
		      \label{f:add-component-request-body}
	      \end{figure}

	      Тело ответа содержит id вновь созданного компонента в случае успеха с
	      кодом ответа 201;

	\item удалить компонент:

	      Запрос: PATCH /api/bots/{botId}/groups/{groupId}/components/{compId}/data.

	      В случае успеха код ответа 204;


	\item изменить данные компонента:

	      Запрос: PATCH /api/bots/{botId}/groups/{groupId}/components/{compId}/data.

	      Тело запроса представлено на рисунке~\ref{f:update-component-data}.


	      \begin{figure}[!ht]
		      \centering
		      \vspace{\toppaddingoffigure}
		      \begin{lstlisting}
{
    "<property name>": "any",
    ...
}
    \end{lstlisting}
		      \caption{Тело запроса при изменении данных компонента}
		      \label{f:update-component-data}
	      \end{figure}

	      В случае успеха код ответа 204;

	\item изменить позицию компонента:

	      Запрос: PATCH /api/bots/{botId}/groups/{groupId}/components/{compId}/position.

	      Тело запроса представлено на рисунке~\ref{f:update-component-position}.


	      \begin{figure}[!ht]
		      \centering
		      \vspace{\toppaddingoffigure}
		      \begin{lstlisting}
{
    "x": "integer",
    "y": "integer"
}
    \end{lstlisting}
		      \caption{Тело запроса при изменении позиции компонента}
		      \label{f:update-component-position}
	      \end{figure}

	      В случае успеха код ответа 204;

	\item добавить соединение между компонентами:

	      Запрос: POST /api/bots/{botId}/groups/{groupId}/connections.

	      Тело запроса представлено на рисунке~\ref{f:add-connection}.

	      \begin{figure}[!ht]
		      \centering
		      \vspace{\toppaddingoffigure}
		      \begin{lstlisting}[
        xleftmargin=.1\textwidth,
        xrightmargin=.1\textwidth
    ]
{
    "sourceComponentId": "integer",
    "sourcePointName": "string",
    "targetComponentId": "integer",
    "relativePointPosition": {
        "x": "integer",
        "y": "integer"
    }
}
    \end{lstlisting}
		      \caption{Тело запроса при добавлении соединения между компонентами}
		      \label{f:add-connection}
	      \end{figure}

	      В случае успеха код ответа 204;

	\item добавить соединение между компонентами:

	      Запрос: DELETE /api/bots/{botId}/groups/{groupId}/connections.

	      Тело запроса представлено на рисунке~\ref{f:delete-connection}.

	      \begin{figure}[!ht]
		      \centering
		      \vspace{\toppaddingoffigure}
		      \begin{lstlisting}[
        xleftmargin=.1\textwidth,
        xrightmargin=.1\textwidth
    ]
{
    "sourceComponentId": "integer",
    "sourcePointName": "string"
}
    \end{lstlisting}
		      \caption{Тело запроса при удалении соединения между компонентами}
		      \label{f:delete-connection}
	      \end{figure}

	      В случае успеха код ответа 204.

\end{itemize}


Если пользователь ввел невалидные данные в любой из приведенных
выше запросов, будет код ответа – 422.

При обращении к API каждый запрос должен содержать следующий
заголовок: “Authorization: Bearer <token>”.

