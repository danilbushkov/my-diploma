
\subsubsection{Реализация модуля компонентов}


Модуль компонентов содержит реализацию компонентов конструктора, а также
предоставляет интерфейсы для работы с ними

\paragraph{Интерфейсы модуля}

Компоненты имеют следующую общую структуру, представленную на
рисунке~\ref{f:comp-struct}.


\begin{figure}[ht]
	\centering
	\vspace{\toppaddingoffigure}
	\begin{lstlisting}
{
    id: number,
    type: string,
    path: string,
    data: {
        <field1>: any,
        ...
    },
    outputs: {
        <field1>: number,
        <field2>: number,
        ...
    }
}

    \end{lstlisting}
	\caption{Общая структура компонентов}
	\label{f:comp-struct}
\end{figure}

Структура содержит следующие поля:
\begin{itemize}
	\item поле id идентифицирует компонент;
	\item поле type содержит тип компонента в виде строки;
	\item поле path предназначен для хранения пути данных для контекста,
	      оно будет использоваться в зависимости от компонента: по нему будут или
	      записываться данные, или считываться для последующего использования;
	\item поле data содержит данные, которые будут использоваться
	      компонентом. Данные индивидуальны для каждого компонента;
	\item поле outputs содержит выходы компонента - id следующего
	      возможного компонента. Часть полей общие, а остальные индивидуальны.

\end{itemize}
Общее поле outputs для всех компонентов - id следующего компонента.
Для некоторых компонентов задаётся только во время выполнения.

Если в поле outputs не определен какой-либо выход, который требуется
для какого-либо компонента, ему присваивается пустое значение.

Объект компонента создается функцией, представленной на рисунке~\ref{f:new-component}.

\begin{figure}[ht]
	\centering
	\vspace{\toppaddingoffigure}
	\begin{lstlisting}[
        language=Go,
        xleftmargin=.1\textwidth,
        xrightmargin=.1\textwidth
    ]
func NewComponentFromJSON(
    tp ComponentType, 
    jsonData []byte
    ) (Component, error)
    \end{lstlisting}
	\caption{Функция создания компонента}
	\label{f:new-component}
\end{figure}

Функция принимает тип компонента в виде строки и данные json,
представленные в виде массива байтов.

Каждый компонент имеет свою логику выполнения. Выполнением
компонентов занимается объект Executor (Исполнитель).

Исполнитель состоит из контекста и интерфейса ввода-вывода.
Создание объекта исполнителя происходит с помощью функции,
представленной на рисунке~\ref{f:new-executor}.


\begin{figure}[ht]
	\centering
	\vspace{\toppaddingoffigure}
	\begin{lstlisting}[
        language=Go,
        xleftmargin=.1\textwidth,
        xrightmargin=.1\textwidth
    ]
func NewExecutor(ctx *Context, io IO) *Executor 
    \end{lstlisting}
	\caption{Функция создания исполнителя}
	\label{f:new-executor}
\end{figure}

Исполнитель имеет лишь один метод, представленный на рисунке~\ref{f:execute}.


\begin{figure}[ht]
	\centering
	\vspace{\toppaddingoffigure}
	\begin{lstlisting}[
        language=Go,
        xleftmargin=.1\textwidth,
        xrightmargin=.1\textwidth
    ]
func (e *Executor) Execute(
        component Component) (*int64, error)
    \end{lstlisting}
	\caption{Метод выполнения компонентов}
	\label{f:execute}
\end{figure}


В качестве параметра выступает интерфейс компонента. В данном
методе происходит выполнение логики компонента. Компонент выполняется
в зависимости от реализации интерфейса определенного вида компонента.
Результатом является id следующего компонента.

Передача данных между компонентами происходит через Context. Так
как структура данных контекста изначально не определена, внутри Context
находятся данные неопределенного типа.

Объект Context создаётся из JSON методом, представленным на рисунке~\ref{f:new-context}.

\begin{figure}[ht]
	\centering
	\vspace{\toppaddingoffigure}
	\begin{lstlisting}[
        language=Go,
        xleftmargin=.05\textwidth,
        xrightmargin=.05\textwidth
    ]
func NewContextFromJSON(jsonData []byte) (*Context, error) 
    \end{lstlisting}
	\caption{Функция создания контекста}
	\label{f:new-context}
\end{figure}

Интерфейс ввода-вывода используется при выполнении компонентов.
Он позволяет взаимодействовать со средой, в которой выполняются
компоненты.

Интерфейс ввода-вывода имеет вид, представленный на рисунке~\ref{f:io-interface}.


\begin{figure}[ht]
	\centering
	\vspace{\toppaddingoffigure}
	\begin{lstlisting}[
        language=Go,
        xleftmargin=.08\textwidth,
        xrightmargin=.08\textwidth
    ]
type IO interface {
	PrintText(text string)
	PrintButtons(text string, buttons []*ButtonData)
	ReadText() *string
}
    \end{lstlisting}
	\caption{Интерфейс ввода-вывода}
	\label{f:io-interface}
\end{figure}

Окружение, которое будет использовать компоненты, должна
реализовать данный интерфейс.

Интерфейс состоит из следующих методов:
\begin{itemize}
	\item PrintText - Вывод текста пользователю;
	\item PrintButtons - Вывод кнопок пользователю с указанием текста
	      опроса;
	\item ReadText - Считывание текста от пользователя, если текст не
	      введен, возвращается нулевое значение - nil.
\end{itemize}


\paragraph{Список компонентов}

В модуле были реализованы следующие компоненты:
\begin{itemize}
	\item стартовый компонент;
	\item форматирование;
	\item сообщение;
	\item ввод текста;
	\item условие;
	\item код;
	\item http;
	\item фото;
	\item кнопки.
\end{itemize}


Стартовый компонент служит точкой запуска бота. Его основная задача
- вернуть следующий компонент для выполнения. У компонентов нет данных,
выход лишь один – nextComponentId (id следующего компонента).

Компонент форматирования служит для преобразования текста к
нужному формату посредством вставки необходимых данных. Имеет одно
поле данных – formatString (строку форматирования). В качестве выхода
служит поле nextComponentId.

Для вставки данных из контекста служит
конструкция, представленная на рисунке~\ref{f:format-construct}.


\begin{figure}[ht]
	\centering
	\vspace{\toppaddingoffigure}
	\begin{lstlisting}
\${<getting data from context>}
    \end{lstlisting}
	\caption{Конструкция в тексте, которая позволяет вставлять данные из
		контекста в компоненте форматирования}
	\label{f:format-construct}
\end{figure}


Компонент сообщение отправляет текст пользователю. Отправка
происходит через интерфейс ввода-вывода. Данными служит выводимый
текст – поле text, выход одни – nextComponentId.

Компонент ввода текста ожидает ввода текста от пользователя.
Получение текста происходит через интерфейс ввода-вывода. У компонента
нет данных, выход один – nextComponentId.

Компонент условия проверяет переменную из контекста: если в ней содержится
истинность, то идёт переход на следующий компонент, если ложь - переход происходит
по другому выходу.

Компонент кода позволяет пользователю расширить функционал компонентов путём
написания небольших скриптов.

Компонент http позволяет обращаться к другим сервисам через соответствующий протокол.
В качестве параметров выступают адрес, метод, заголовки и тело запроса.

Компонент фото отправляет картинку пользователю. Параметрами являются название и картинка,
представляющая собой массив байт.

Компонент кнопок выводит набор кнопок для пользователя. Вывод
кнопок происходит через интерфейс ввода-вывода.
Содержит данные, представленные на рисунке~\ref{f:button-component-data}.

Выходы компонента кнопок представлены на рисунке~\ref{f:button-component-outputs}.
Имена выходов совпадают с элементами поля buttons в data.

\begin{figure}[ht]
	\centering
	\vspace{\toppaddingoffigure}
	\begin{lstlisting}[
        xleftmargin=.15\textwidth,
        xrightmargin=.15\textwidth
        ]
type Button = {
    text: string;
}

data: {
    text: string,
    buttons: {
        <numeric field name>: Button,
        ...
    }
}
    \end{lstlisting}
	\caption{Данные компонента кнопок}
	\label{f:button-component-data}
\end{figure}

\begin{figure}[!ht]
	\centering
	\vspace{\toppaddingoffigure}
	\begin{lstlisting}
outputs: {
    <numeric field name>: number,
    ...
}
    \end{lstlisting}
	\caption{Выходы компонента кнопок}
	\label{f:button-component-outputs}
\end{figure}

\newpage


