
\subsection{Формат передаваемых данных}

Для взаимодействия клиентской и серверной частей требуется
выбрать формат передаваемых данных.

Рассмотрим два формата передачи данных: XML и JSON.

JSON – текстовый формат обмена данными, основанный на JavaScript.
Применяется в веб-приложениях как для обмена данными между браузером и
сервером, так и между серверами (программные HTTP-сопряжения).

Преимущество данного формата:
\begin{itemize}
	\item удобен для чтения человеком;
	\item поскольку JSON является подмножеством синтаксиса языка
	      JavaScript, то он может быть быстро десериализован стандартной библиотекой
	      этого языка на стороне браузера.
\end{itemize}

XML - формат документа, в котором использованы теги для определения
объектов и их атрибутов. Используется для формирования структуры
документа и как формат обмена данными.


Преимущества данного формата:
\begin{itemize}
	\item удобен для создания структуры документа;
	\item расширяем.
\end{itemize}
Минусы:
\begin{itemize}
	\item трудно читаем по сравнению с вышеописанным форматом;
	\item из-за тегов избыточен при обмене данных.
\end{itemize}

Из-за удобочитаемости и нативной поддержки в JavaScript для
взаимодействия клиента и сервера был выбран формат JSON.
