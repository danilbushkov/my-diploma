

\subsection{Расширенное техническое задание}

В данном подразделе представлено техническое задание на разработку
конструктора Telegram-ботов.

\subsubsection{Краткая характеристика области применения}

Программа предназначена для создания
Telegram-ботов, которые будут удовлетворять
потребности клиентов в создании их бизнес-решений.


\subsubsection{Основание для разработки}

Функциональным назначением программы является предоставление
клиентам возможности создания Telegram-ботов при помощи визуального редактора и
без глубоких знаний языков программирования.

Программа должна эксплуатироваться на серверах клиента.
Для использования конечному пользователю предъявляются требования знания процесса
развёртывания серверных приложений.


\subsubsection{Требования к структуре}

Конструктор должен состоять из двух частей: серверной и клиентской.
Серверная часть должна обеспечивать основной функционал конструктора.
Клиентская часть предоставляет собой удобный пользовательский интерфейс.

\subsubsection{Требования к серверной части конструктора}

В подпунктах данного подраздела описываются требования к серверной части конструктора.

\paragraph{Требования к функциональным характеристикам}

Серверная часть конструктора должна предоставлять программный
интерфейс, который обеспечивает выполнение следующих функций:
\begin{itemize}
	\item регистрация пользователей;
	\item аутентификация пользователей;
	\item создание ботов;
	\item вывод списка ботов;
	\item запуск и остановку ботов;
	\item добавлять компоненты;
	\item удалять компоненты;
	\item редактировать содержимое компонента;
	\item соединять компоненты;
	\item обслуживать запросы пользователей от запущенных ботов.
\end{itemize}

\paragraph{Требования к предоставляемому программному интерфейсу}

Программный интерфейс, который предоставляется серверной частью
конструктора должен удовлетворять следующим требованиям:
\begin{itemize}
	\item должен быть прост и интуитивен;
	\item должен быть надежен и доступен;
	\item должен включать возможность аутентификации и авторизации;
	\item должен обрабатывать возможные ошибки при запросе
	      пользователя.
\end{itemize}

\paragraph{Требования к параметрам технических и программных средств}

В состав технических средств должен входить компьютер, включающий
в тебя:
\begin{itemize}
	\item 64-разрядный процессор с тактовой частотой не менее 1.0 ГГц;
	\item не менее 4 гигабайт оперативной памяти;
	\item не менее 1 гигабайт свободного дискового пространства;
	\item сетевую карту.
\end{itemize}

Также для работы севера требуется предустановленная операционная
система на базе ядра Linux: Ubuntu 18.04 или старше, Debian 10 или старше.

\subsubsection{Требования к клиентской части конструктора}

В подпунктах данного подраздела описываются требования к клиентской части конструктора.


\paragraph{Требования к функциональным характеристикам}

Клиентская часть конструктора должна иметь возможность
формировать и отправлять данные и запросы для выполнения следующих
функций:
\begin{itemize}
	\item регистрация пользователей;
	\item аутентификация пользователей;
	\item создание ботов;
	\item вывод списка ботов;
	\item запуск и остановку ботов;
	\item редактирование ботов.
\end{itemize}

Для работы с содержимым ботов клиентская часть должна содержать
визуальный редактор. Редактор предоставляет пользовательский интерфейс
для выполнения следующих функций:
\begin{itemize}
	\item добавление компонентов;
	\item удаление компонентов;
	\item редактирование содержимого компонентов;
	\item соединение компонентов.
\end{itemize}


\paragraph{Требования к пользовательскому интерфейсу}

Интерфейс конструктора Telegram ботов должен состоять из страниц,
содержащих разные визуальные элементы, которые предоставляют
пользователям возможность взаимодействовать с конструктором.

Также интерфейс должен обеспечивать
наглядное, интуитивно понятное представление.

\paragraph{Требования к клиентскому программному обеспечению}

Клиентская часть конструктора Telegram-ботов должна быть доступна для
полнофункционального использования с помощью следующих браузеров:
\begin{itemize}
	\item Edge 88.0 и выше;
	\item Opera 43.0 и выше;
	\item Mozilla Firefox 55.0;
	\item Google Chrome 64.0 и выше.
\end{itemize}

\subsubsection{Стадии разработки}

Разработка должна быть проведена в следующих стадиях:
\begin{itemize}
	\item разработка технического задания;
	\item проектирование структуры серверной части конструктора;
	\item проектирование структуры клиентской части конструктора;
	\item программная реализация серверной части конструктора;
	\item программная реализация клиентской части конструктора.
\end{itemize}

