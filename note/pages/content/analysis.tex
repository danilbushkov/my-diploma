
\section{Анализ предметной области}


На данном этапе работы необходимо рассмотреть функции
конструкторов Telegram-ботов,
провести обзор существующих на данный момент аналогов, рассмотреть их возможности,
выявить их
недостатки и обосновать актуальность разработки нового конструктора.

\subsection{Telegram-боты и конструкторы}

Боты в мессенджере Telegram становятся все более популярными и
число их пользователей постоянно растет. Они помогают пользователям
выполнять типичные рутинные действия в автоматизированном режиме,
значительно упрощая им жизнь. Для владельцев же самих ботов они стали
незаменимыми помощниками в работе.

Telegram-боты имеют множество плюсов, таких как:
\begin{itemize}
	\item круглосуточный доступ;
	\item моментальный ответ на запрос пользователя;
	\item удобство использования, интуитивно понятный интерфейс;
	\item не требуется установка дополнительных программ,
	      общение с ботом ведется через мессенджер.
\end{itemize}

Telegram-бот используют в коммерческой деятельности для следующих сфер и задач:
\begin{itemize}
	\item развлечения;
	\item поиск и обмен файлами;
	\item предоставление новостей;
	\item утилиты и инструменты;
	\item интеграция с другими сервисами;
	\item осуществление онлайн-платежей.
\end{itemize}


С популярностью ботов стали появляться все больше различных
конструкторов, которые позволяют
без наличия специальных знаний и
навыков создать своего бота всего в несколько кликов.

Конструктором называется NoCode инструмент, который
предназначен для
быстрого создания ботов без знания каких-либо языков программирования.
Иными словами, весь процесс создания – это нажатие тех или иных кнопок и
ввода текста (например, название кнопки, текст сообщения и т.д.).

Первое предназначение – упрощение работы. Ведь не все обладают
глубокими знаниями и навыками программирования. Когда боты только
появились, их могли разрабатывать только программисты, обладающие
соответствующим опытом и навыками.

Помимо того, что конструкторы позволяют расширить аудиторию,
способную создавать Telegram-ботов, они экономят время разработчикам. При
наличии конструктора нет необходимости разрабатывать каждый раз
отдельное приложение для выполнения типовых
задач, так как конструктор предоставляет необходимый
набор инструментов для быстрого создания бота
без необходимости писать код.

Но у конструкторов есть некоторые ограничения, например, при их
использовании нельзя выйти за рамки возможностей самого конструктора, а
также при выходе нового функционала Telegram API, его реализация в
конструкторе происходит с некоторой задержкой. Кроме того, боты,
реализованные с помощью NoCode решения обычно менее производительные,
чем их аналоги, написанные языке программирования.



\subsection{Обзор аналогов}

В подпунктах данного раздела рассматриваются существующие аналоги.
В качестве рассматриваемых аналогов были выбраны приложения,
реализующие функционал создания Telegram ботов с помощью конструктора.


\subsubsection{Бот-платформа «ManyBot»}


Один из наиболее популярных конструкторов. Работает внутри мессенджера Telegram.
Он бесплатный и прост в использовании.
Бот предоставляет создание бота с такими возможностями:
\begin{itemize}
	\item отправка сообщений;
	\item создание меню;
	\item автопостинг из VK, Twitter, YouTube;
	\item поддержка нескольких языков.
\end{itemize}

Минусы конструктора состоят в том, что нет администрирования
бота за пределами мессенджера, наличие рекламного сообщения в
созданном боте, малое количество компонентов и их модификаций,
а также отсутствие статистических данных по созданному боту.

\subsubsection{Конструктор Telegram ботов «Puzzlebot»}


Данный конструктор имеет намного больше возможностей, чем предыдущий сервис:
удобный личный кабинет, интуитивный интерфейс, имеет намного больше компонентов,
позволяющих реализовывать сложных ботов.

Минусы данного решения в том, что на бесплатном тарифе
можно создать лишь одного бота и настроить до 15 команд, а также
количество участников бота ограничено 150 пользователями.

\subsubsection{Конструктор ботов «Botmother»}

Довольно мощный сервис по созданию ботов, который имеет удобный интерфейс
для администрирования и создания ботов, предоставляет много различных компонентов,
возможность просмотра статистики созданного бота.

В бесплатном тарифе предоставляет лишь создание 10 тестовых ботов с ограниченным
функционалом.

\subsubsection{Сравнение аналогов}

В таблице \ref{t:comp-a} представлено сравнение вышеперечисленных аналогов.

\begin{table}[ht]
	\Large
	\caption{Сравнение аналогов}
	\label{t:comp-a}
	\centering
	\begin{tabularx}{\textwidth}{|X|c|c|c|}
		\hline
		Критерии \textbackslash\ Аналоги & Manybot & Puzzlebot & Botmother \\
		\hline
		Удобный доступ для администрирования бота
		                                 & нет     & да        & да        \\
		\hline
		Изменение порядка вызовов компонентов
		                                 & нет     & да        & да        \\
		\hline
		Нет ограничений на использования компонентов
		                                 & да      & да        & нет       \\
		\hline
		Отсутствие рекламы
		                                 & нет     & нет       & да        \\
		\hline
	\end{tabularx}
	\vspace{\bottompaddingoftable}
\end{table}


Как видно из таблицы \ref{t:comp-a}
существующие решения имеют ряд недостатков.
Также конструкторы больше ориентированы на получение прибыли и
ограничивают функционал для бесплатного использования.

Учитывая недостатки рассмотренных аналогов разрабатываемое приложение должно обеспечивать следующий функционал:

\begin{itemize}

	\item возможность создания ботов с помощью конструктора;
	\item возможность администрирования бота;
	\item возможность изменения порядка вызовов компонентов;
	\item отсутствие ограничений на использование компонентов;
	\item отсутствие рекламы.

\end{itemize}

\subsection{Актуальность разработки}

Telegram боты являются функциональными инструментами для многих пользователей,
однако для их разработки зачастую требуются навыки программирования,
что усложняет их внедрение в бизнес-процессы компаний.
Конструкторы Telegram-ботов по большей части решают данную проблему,
предоставляя пользователям удобный интерфейс для создания ботов под конкретные задачи.

Большинство компаний предоставляют ограниченный
функционал конструктора, а для расширения их возможности требуют дополнительную
плату, что не всегда выгодно для конечного пользователя.
Поэтому было принято решение о создании нового конструктора, который исключает
вышеперечисленные недостатки, предоставляя пользователям свободную платформу для
создания ботов.




\subsection{Расширенное техническое задание}

В данном разделе представлено техническое задание на разработку
конструктора Telegram-ботов.

\subsubsection{Краткая характеристика области применения}

Программа предназначена для создания
Telegram-ботов, которые будут удовлетворять
потребности клиентов в создании их бизнес-решений.


\subsubsection{Основание для разработки}

Функциональным назначением программы является предоставление
клиентам возможности создания Telegram-ботов при помощи визуального редактора и
без глубоких знаний языков программирования.

Программа должна эксплуатироваться на серверах клиента.
Для использования конечному пользователю предъявляются требования знания процесса
развёртывания серверных приложений.


\subsubsection{Требования к структуре}

Конструктор должен состоять из двух частей: серверной и клиентской.
Серверная часть должна обеспечивать основной функционал конструктора.
Клиентская часть предоставляет собой удобный пользовательский интерфейс.

\subsubsection{Требования к серверной части конструктора}

В подпунктах данного раздела описываются требования к серверной части конструктора.

\paragraph{Требования к функциональным характеристикам}

Серверная часть конструктора должна предоставлять программный
интерфейс, который обеспечивает выполнение следующих функций:
\begin{itemize}
	\item регистрация пользователей;
	\item аутентификация пользователей;
	\item создание ботов;
	\item вывод списка ботов;
	\item запуск и остановку ботов;
	\item добавлять компоненты;
	\item удалять компоненты;
	\item редактировать содержимое компонента;
	\item соединять компоненты;
	\item обслуживать запросы пользователей от запущенных ботов.
\end{itemize}

\paragraph{Требования к предоставляемому программному интерфейсу}

Программный интерфейс, который предоставляется серверной частью
конструктора должен удовлетворять следующим требованиям:
\begin{itemize}
	\item должен быть прост и интуитивен;
	\item должен быть надежен и доступен;
	\item должен включать возможность аутентификации и авторизации;
	\item должен обрабатывать возможные ошибки при запросе
	      пользователя.
\end{itemize}

\paragraph{Требования к параметрам технических и программных средств}

В состав технических средств должен входить компьютер, включающий
в тебя:
\begin{itemize}
	\item 64-разрядный процессор с тактовой частотой не менее 1.0 ГГц;
	\item не менее 4 гигабайт оперативной памяти;
	\item не менее 1 гигабайт свободного дискового пространства;
	\item сетевую карту.
\end{itemize}

Также для работы севера требуется предустановленная операционная
система на базе ядра Linux: Ubuntu 18.04 или старше, Debian 10 или старше.

\subsubsection{Требования к клиентской части конструктора}

В подпунктах данного раздела описываются требования к клиентской части конструктора.


\paragraph{Требования к функциональным характеристикам}

Клиентская часть конструктора должна иметь возможность
формировать и отправлять данные и запросы для выполнения следующих
функций:
\begin{itemize}
	\item регистрация пользователей;
	\item аутентификация пользователей;
	\item создание ботов;
	\item вывод списка ботов;
	\item запуск и остановку ботов;
	\item редактирование ботов.
\end{itemize}

Для работы с содержимым ботов клиентская часть должна содержать
визуальный редактор. Редактор предоставляет пользовательский интерфейс
для выполнения следующих функций:
\begin{itemize}
	\item добавление компонентов;
	\item удаление компонентов;
	\item редактирование содержимого компонентов;
	\item соединение компонентов.
\end{itemize}


\paragraph{Требования к пользовательскому интерфейсу}

Интерфейс конструктора Telegram ботов должен состоять из страниц,
содержащих разные визуальные элементы, которые предоставляют
пользователям возможность взаимодействовать с конструктором.

Также интерфейс должен обеспечивать
наглядное, интуитивно понятное представление.

\paragraph{Требования к клиентскому программному обеспечению}

Клиентская часть конструктора Telegram-ботов должна быть доступна для
полнофункционального использования с помощью следующих браузеров:
\begin{itemize}
	\item Edge 88.0 и выше;
	\item Opera 43.0 и выше;
	\item Mozilla Firefox 55.0;
	\item Google Chrome 64.0 и выше.
\end{itemize}

\subsubsection{Стадии разработки}

Разработка должна быть проведена в следующих стадиях:
\begin{itemize}
	\item разработка технического задания;
	\item проектирование структуры серверной части конструктора;
	\item проектирование структуры клиентской части конструктора;
	\item программная реализация серверной части конструктора;
	\item программная реализация клиентской части конструктора.
\end{itemize}




\section*{Выводы}

В данном разделе был проведен анализ предметной области и осуществлен обзор аналогов.
Было выявлено, что многие решения имеют ряд недостатков,
таких как ограничения на использование компонентов и показ рекламы.
Таким образом, данная тематика
и разработка конструктора Telegram-ботов является актуальной.

Из рассмотренных аналогов были выявлены требуемые
функциональные возможности разрабатываемого продукта.
Было составлено расширенное техническое задание.
Определены основные требования к разрабатываемому конструктору Telegram-ботов, такие как
требования к структуре, требования к функциональным характеристикам,
требования к интерфейсу и к
клиентскому аппаратному и программному обеспечению.
Также выделены основные стадии разработки.


