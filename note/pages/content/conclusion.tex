\newpage

\csection{Заключение}

В ходе выполнения курсового проекта была выявлена структура
серверной части для конструктора Telegram ботов, предоставляющая собой
набор сервисов. Сервисы предоставляют программный интерфейс для
управления конструктором.

Для авторизации и реализации компонентов были выделены отдельные
модули. Модуль компонентов был также разбит на подмодули, которые
предоставляют интерфейсы для исполнения компонентов.

Был определен алгоритм обработки запросов пользователей бота,
который заключается в выполнении компонентов и сохранения текущего
состояния контекста пользователя бота после обработки.

Для разработки были выбраны оптимальные инструменты и определен
формат передаваемых данных, также разработан интерфейс для работы с
сервисами, который работает по протоколу HTTP.

По разработанной структуре была реализована серверная часть
конструктора Telegram ботов.


В ходе выполнения курсового проекта была выявлена структура
клиентской части для конструктора Telegram ботов, предоставляющая собой
набор страниц, с помощью которых пользователь может взаимодействовать с
конструктором.

Представлена модульная и компонентная структуры визуального
редактора. Модульная структура позволила выделить определенные
функциональные блоки редактора – набор структур и функций, которые
ответственны за определенную часть редактора; компонентная структура
помогла представить редактор как набор связанных компонентов в виде
дерева.

Была разработана диаграмма состояний для клиентской части
конструктора и редактора ботов, показывающая их поведение при действиях
пользователя.

Для разработки были выбраны оптимальные инструменты, определены
формат и структуры передаваемых данных.

По разработанной структуре была реализована клиентская часть
конструктора Telegram ботов.
