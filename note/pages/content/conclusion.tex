\newpage

\csection{Заключение}

В ходе выполнения выпускной квалификационной работы был проведен анализ
предметной области и обзор существующих аналогов конструктора
Telegram-ботов. У рассмотренных аналогов были выявлены основные
недостатки - ограниченность функционала при бесплатном использовании
и показ рекламы. Также данные решения обладают закрытым исходным кодом,
что не гарантирует приватность данных пользователей.
Поэтому было принято решение о создании конструктора, который был бы лишен
данных недостатков.

Из рассмотренных аналогов были выявлены требуемые функциональные
возможности разрабатываемого продукта.
Исходя из этих возможностей, было составлено расширенное техническое задание.

Была разработана общая структура конструктора, включающая в себя
серверную и клиентскую части. Серверная часть обеспечивает выполнение
основного функционала конструктора, а клиентская предоставляет
пользователям удобный интерфейс для работы с ним.
Серверная часть была разбита на ряд сервисов, которые инкапсулируют
определенную часть функционала конструктора и предоставляют доступ
к этим функциям через определенный программный интерфейс.
Этими сервисами являются сервис пользователей,
сервис ботов и сервис, обслуживающий ботов.

Для авторизации и реализации компонентов были выделены отдельные
модули. Модуль компонентов был также разбит на подмодули, которые
предоставляют интерфейсы для исполнения компонентов.
С помощью реализации интерфейсов подмодуля ввода-вывода, возможно выполнение
компонентов бота в разных мессенджерах.

Были определены алгоритмы обработки запросов пользователей сервиса ботов
и пользователей бота.
Алгоритм обработки запросов пользователей бота
заключается в выполнении компонентов и сохранения текущего
состояния контекста пользователя бота после обработки. Данный алгоритм
обеспечивает работу ботов.

Интерфейс клиентской части был разбит на набор страниц, где
находятся основные визуальные объекты для взаимодействия с конструктором.
Для содержимого каждой страницы был разработан свой шаблон.

Представлены модульная и компонентная структуры визуального
редактора. Модульная структура позволила выделить определенные
функциональные блоки редактора – набор структур и функций, которые
ответственны за определенную часть редактора; компонентная структура
помогла представить редактор как набор связанных компонентов в виде
дерева.

Была разработана диаграмма состояний для клиентской части
конструктора и редактора ботов, показывающая их поведение при действиях
пользователя.

Для разработки были выбраны оптимальные инструменты и определен
формат передаваемых данных, также был разработан интерфейс для обеспечения
взаимодействия с сервисами, который работает по протоколу HTTP.

По разработанным структурам и алгоритмам был реализован конструктор Telegram-ботов.
Исходный код разрабатываемого приложения выложен в открытый доступ на сервисе GitHub.
Таким образом, каждый желающий может ознакомиться с ним и развернуть конструктор
у себя, что предотвращает использование данных пользователей в нежелательных целях.
Также пользователь сможет расширить возможности конструктора: добавить новые компоненты
или ввести новые сервисы, которые будут обеспечивать работу ботов в других мессенджерах.


